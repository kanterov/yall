\documentclass[a4paper,12pt,oneside]{article} 

\usepackage[utf8]{inputenc}
\usepackage[russian]{babel} 
\usepackage{cmap} % only with pdftex
%\usepackage{indentfirst}
\usepackage{latexsym}

\usepackage[unicode=true]{hyperref}
\usepackage{listings}

\hypersetup{
    colorlinks,
    citecolor=black,
    filecolor=black,
    linkcolor=black,
    urlcolor=black
}

\lstdefinelanguage{ebnf} {
  sensitive=false,
  morecomment=[s]{<}{>},
  morestring=[b]",
}


\author{Кантеров Глеб}
\title{YALL reference}



\begin{document}

\begin{titlepage}
\end{titlepage}

\tableofcontents
\newpage

\section{Описание синтаксиса}
\lstinputlisting[language=ebnf, breaklines=true, breakatwhitespace=true]{syntax.ebnf}

\section{Основные принципы}

Код состоит из выражений, которые вычисляются последовательно. Выражение состоит из фукнции, 
которая принимает некоторое, больше или равное 1, число параметров, которые могут являться
константами, переменными или другими выражениями, в этом случае для вычисления функции требуется
вычислить это выражение, это делается последовательно слева направо. Исключениями являются 
\textbf{any} и \textbf{do}, их поведение описано далее. При работе с переменными они должны 
быть определены.

\section{Функции языка}

\subsection{list}
\textbf{Параметры:} принимает неограниченное число параметров любого типа \\
\textbf{Возвращает:} значение последнего параметра \\
\textbf{Использование:} используется главным образом для вычисления последовательности функций \\
\textbf{Пример:} 
\begin{lstlisting}[language=lisp]
(list 
    (list 1) 
    (list 1 2)
) 
\end{lstlisting}

\subsection{any}
\textbf{Параметры:} принимает 2 параметра любого типа \\
\textbf{Возвращает:} равновероятно выбирает 1 или 2 параметр, вычисляет его и возвращает его значение \\
\textbf{Пример:} 
\begin{lstlisting}[language=lisp]
(any 1 
    (any 2 
        (any 3 4)
))
\end{lstlisting}

\subsection{do}
\textbf{Параметры:} принимает 2 параметра любого типа \\
\textbf{Возвращает:} вычисляет первый параметр $n \in [ 0, \inf ] $ раз и возвращает его,
    если $n = 0$ возвращает значение второго параметра \\
\textbf{Пример:} 
\begin{lstlisting}[language=lisp]
(do 
    (def 
        (* n 2)
        n
    ) 
    (def 
        -1 
        n
    )
)
\end{lstlisting}


\subsection{Арифметические и логические: +, -, /, *, leq, geq, le, gr}
\textbf{Параметры:} принимает 2 целых параметра \\
\textbf{Возвращает:} вычисляет арифметическое или логическое выражение, в случае логического 0 это false,
    не 0 - true. При делении на ноль функция может принимать любое целое значение. \\
\textbf{Пример:} 
\begin{lstlisting}[language=lisp]
(def
    (+
        (*  4
            5
        )
        (/  8
            2
        )
    )
    n
)
\end{lstlisting}


\subsection{array}
\textbf{Параметры:} принимает 2 параметра любого типа \\
\textbf{Возвращает:} объект типа массив, отображение из первого типа данных во второй с уже заданным соответствием
    между параметрами. (отображение ключ $\rightarrow$ значение) \\
\textbf{Пример:} Вернется массив с уже заданным 1337 $\rightarrow$ 42
\begin{lstlisting}[language=lisp]
(array
    1337
    42
)
\end{lstlisting}

\subsection{get} 
\textbf{Параметры:} принимает 2 параметра: параметр типа ключа и массив \\
\textbf{Возвращает:} значение, соответствующее заданому ключу, если его нет в массиве, то все что угодно \\
\textbf{Пример:} Вернется 42
\begin{lstlisting}[language=lisp]
(get
    1337
    (array
        1337
        42
    )
)
\end{lstlisting}

\subsection{put} 
\textbf{Параметры:} принимает 3 параметра: параметр типа ключа, параметр типа значение и идентификатор
    переменной-массив \\
\textbf{Возвращает:} массив-копию с измененным отображением ключ $\rightarrow$ значение, где добавлена 
    переданная как параметры пара \\
\textbf{Пример:} Вернется массив с уже заданными 1337 $\rightarrow$ 42, 42 $\rightarrow$ 1337, 0 $\rightarrow$ -1
\begin{lstlisting}[language=lisp]
(list
    (def
        (array
            1337
            42
        )
        var
    )
    (def 
        (put
            0
            -1
            var
        )
        var
    )
)
\end{lstlisting}

\subsection{?}
\textbf{Параметры:} принимает 1 параметр - целое число \\
\textbf{Возвращает:} 0, в случае если значение параметра равно 1, иначе значение полагается неопределенным и
    данное выражение высчитывается заново после вычислениях всех остальных предшедствующих параметров в функции,
    которой потребовалось вычислить это выражение. Если этой функции нет, т.е оно вычисляет на правах выражения в 
    коде программы, то аналогично все выражения будут вычисленны заново. \\
\textbf{Пример:} После завершения вычислений n=42 
\begin{lstlisting}[language=lisp]
(list
    (def 1 n)
    (do
        (def
            (+ n 1)
            n
        )
        0
    )
    (? (eq n 42))
)
\end{lstlisting}

\subsection{def} 
\textbf{Параметры:} принимает 2 параметра одинакового типа, второй должен быть идентификатором переменной 
    или ссылкой на нее \\
\textbf{Возвращает:} ссылку на второй параметр, создается копия первого параметра и устанавливается 
    соответствие идентификатор второго параметра $\leftrightarrow$ эта переменная. \\
\textbf{Пример:} 
\begin{lstlisting}[language=lisp]
(def
    (array
        1337
        42
    )
    myArray
)
\end{lstlisting}

\section{Реализация сортировки массива}
\lstinputlisting[language=lisp, breaklines=true, breakatwhitespace=true]{sort.yall}


\end{document}
